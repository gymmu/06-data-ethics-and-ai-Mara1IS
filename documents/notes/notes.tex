\documentclass{article}

\usepackage[ngerman]{babel}
\usepackage[utf8]{inputenc}
\usepackage[T1]{fontenc}
\usepackage{hyperref}
\usepackage{csquotes}

\usepackage[
    backend=biber,
    style=apa,
    sortlocale=de_DE,
    natbib=true,
    url=false,
    doi=false,
    sortcites=true,
    sorting=nyt,
    isbn=false,
    hyperref=true,
    backref=false,
    giveninits=false,
    eprint=false]{biblatex}
\addbibresource{../references/bibliography.bib}

\title{Notizen zum Projekt Data Ethics}
\author{Mara Schöni}
\date{\today}

\begin{document}
\maketitle

\abstract{
In der Folgenden Arbeit geht es um KI und ihrer Auswirkung auf die Ethik und die folgen auf unsere Gesellschaft.... noch nicht fertig
}

\tableofcontents

\section{Einleitung zur KI}
\label{sec:ai}

Was ist KI wie wird sie genutzt und wie Funktioniert die Künstliche Intelligenz?
Künstliche Intelligenz (KI) ist vereinfacht gesagt die Fähigkeit einer Maschine, menschliche Fähigkeiten wie logisches Denken, Lernen, Planen und Kreativität zu imitieren.
Ob ChatGPT, Google Translate,bei der Entsperrung von Smartphones oder bei der Identifikation von Personen in Überwachungskameras hinter all diesen alltagsbeispielen und noch hinter vielem mehr steckt die Künstliche Intelligenz. Die KI ein sehr komplexes und grosses Thema. Die KI muss zuerst einmal auf ihre Aufgabe trainiert werden, wie trainiert man jedoch eine Künstliche Intelligenz?

Wie wird die KI trainiert?
Das Training von Künstlicher Intelligenz (KI) beginnt mit der Sammlung großer Mengen relevanter Daten, etwa Röntgenbilder für medizinische Anwendungen. Diese Daten werden bereinigt und in ein nutzbares Format gebracht. Dann wird ein passendes Machine-Learning-Modell ausgewählt und mit den Daten trainiert, um Muster zu erkennen.

Das Modell wird anschließend mit Testdaten überprüft, um seine Genauigkeit zu testen. Danach werden die Modellparameter optimiert, um die Leistung zu verbessern. Wenn das Modell gut funktioniert, wird es in eine reale Umgebung überführt, wo es echte Daten verarbeitet. Schließlich wird das Modell kontinuierlich überwacht und bei Bedarf mit neuen Daten nachtrainiert.

Dieser Prozess ermöglicht es dem KI-Modell, Aufgaben effizient und genau zu erledigen. 

\section{Fragestellung}

In meinem Projekt über die KI und Ethik habe ich mich auf die Fragestellung: "Ist die KI für die Menscheit eine Gefahr oder eine Hilfe?" bezogen. Um diese Komplexe und eher allgemeine Frage einigermassen zu beantworten, bin ich in meiner Arbeit auf verschiedene Aspekte eingegangen.


\printbibliography

\end{document}
